\chapter{The integers}

We construct the integers starting from the natural numbers $\Z$. Since \texttt{lean} already has a
type called $\Z$, we define a new type called $\MyInt$ that will be another definition of the integers.
$\MyInt$ will be a quotient of a type called $\MyPreint$.

\begin{definition}
    \label{def:MyPreint}
    \lean{MyPreint}
    \leanok
    Let $\MyPreint$ be $\N \times \N$
\end{definition}

\begin{definition}
    \label{def:R}
    \lean{MyPreint.R}
    \leanok
We define a relation $R$ on $\MyPreint$ as follows: $(a,b)$ and $(c, d)$ are related if and only if
\[
a + d = b + c
\]
\end{definition}

\begin{lemma}
$R$ is a reflexive relation.
    \label{lemma:reflexive}
    \lean{MyPreint.R_refl}
    \leanok
\end{lemma}
\begin{proof}
    \leanok
    \uses{def:R}
    This follows by commutativity of addition in $\N$.
\end{proof}

\begin{lemma}
$R$ is a symmetric relation.
    \label{lemma:symmetric}
    \lean{MyPreint.R_symm}
    \leanok
\end{lemma}
\begin{proof}
    \leanok
    \uses{def:R}
    This follows by commutativity of addition in $\N$.
\end{proof}

\begin{lemma}
$R$ is a transitive relation.
    \label{lemma:transitive}
    \lean{MyPreint.R_trans}
    \leanok
\end{lemma}
\begin{proof}
    \leanok
    \uses{def:R}
    Let $x$, $y$ and $z$ in $\MyPreint$ such that $x R y$ and $y R z$. We can write $x = (a,b)$ and similarly
    for $y = (c,d)$ and $z = (e,f)$. By assumption we have $a+d=b+c$ and $c+f=d+e$. Adding these equalities we get
    \[
    a+d+c+f=b+c+d+e
    \]
    Since addition on $\N$ is cancellative we get
    \[
    a+f = b + e
    \]
    as wanted.
\end{proof}