\chapter{The rationals}

We can now define the rationals, starting with our copy of the integers $\MyInt$.

We follow a similar path to the one for $\MyInt$.

\section{The \emph{prerationals}}

$\MyRat$ will be a quotient of a type called $\MyPrerat$.

\begin{definition}
    \label{MyPrerat}
    \lean{MyPrerat}
    \leanok
    \uses{MyInt}
    Let $\MyPrerat$ be $\MyInt \times \MyInt \backslash \{0\}$
\end{definition}

\begin{definition}
    \label{MyPrerat.R}
    \lean{MyPrerat.R}
    \leanok
    \uses{MyPrerat}
We define a relation $R$ on $\MyPrerat$ as follows: $(a,b)$ and $(c, d)$ are related if and only if
\[
a * d = b * c
\]
\end{definition}

\begin{lemma}
$R$ is a reflexive relation.
    \label{MyPrerat.R_refl}
    \lean{MyPrerat.R_refl}
    \leanok
\end{lemma}
\begin{proof}
    \leanok
    \uses{MyPrerat.R}
    Exercice.
\end{proof}

\begin{lemma}
$R$ is a symmetric relation.
    \label{MyPrerat.R_symm}
    \lean{MyPrerat.R_symm}
    \leanok
\end{lemma}
\begin{proof}
    \leanok
    \uses{MyPrerat.R}
    Exercice.
\end{proof}

\begin{lemma}
$R$ is a transitive relation.
    \label{MyPrerat.R_trans}
    \lean{MyPrerat.R_trans}
    \leanok
\end{lemma}
\begin{proof}
    \leanok
    \uses{MyPrerat.R}
    Exercice.
\end{proof}

\begin{lemma}
    \label{MyPrerat.R_equiv}
    \lean{MyPrerat.R_equiv}
    \leanok
We have that $R$ is an equivalence relation. From now on, we will write $x \approx y$ for
$x R y$.
\end{lemma}
\begin{proof}
    \leanok
    \uses{MyPrerat.R_refl, MyPrerat.R_symm, MyPrerat.R_trans}
Clear from Lemma~\ref{MyPrerat.R_refl}, Lemma~\ref{MyPrerat.R_symm} and Lemma~\ref{MyPrerat.R_trans}.
\end{proof}

\begin{definition}
    \label{MyPrerat.neg}
    \lean{MyPrerat.neg}
    \leanok
    \uses{MyPrerat.R}
We define an operation, called \emph{negation} on $\MyPrerat$ as follows: the negation of $x = (a,b)$ is
$(-a,b)$:
\[
-x = -(a,b) = (-a,b)
\]
Note that it is automatically well defined (meaning that second component of $(-a,b)$ is different from $0$).
\end{definition}

\begin{lemma}
    \label{MyPrerat.neg_quotient}
    \lean{MyPrerat.neg_quotient}
    \leanok
If $x \approx x'$, then $-x \approx -x'$.
\end{lemma}
\begin{proof}
\uses{MyPrerat.neg}
\leanok
Exercice.
\end{proof}

\begin{definition}
    \label{MyPrerat.add}
    \lean{MyPrerat.add}
    \leanok
    \uses{MyPrerat.R}
We define an operation, called \emph{addition} on $\MyPrerat$ as follows: the addition of $x = (a,b)$
and $y = (b, c)$ is
\[
x + y = (a,b) + (c,d) = (a * d + b * c, b*d)
\]
Do you see why it is well defined?
\end{definition}

\begin{lemma}
    \label{MyPrerat.add_quotient}
    \lean{MyPrerat.add_quotient}
    \leanok
If $x \approx x'$ and $y \approx y'$, then $x + y \approx x' + y'$.
\end{lemma}
\begin{proof}
\uses{MyPrerat.add}
\leanok
Exercice.
\end{proof}

\begin{definition}
    \label{MyPrerat.mul}
    \lean{MyPrerat.mul}
    \leanok
    \uses{MyPrerat.R}
We define an operation, called \emph{multiplication} on $\MyPrerat$ as follows: the multiplication of $x = (a,b)$ and $y = (b, c)$ is
\[
x * y = (a,b) * (c,d) = (a*c, b*d)
\]
\end{definition}

\begin{lemma}
    \label{MyPrerat.mul_quotient}
    \lean{MyPrerat.mul_quotient}
    \leanok
If $x \approx x'$ and $y \approx y'$, then $x * y \approx x' * y'$.
\end{lemma}
\begin{proof}
\uses{MyPrerat.mul}
\leanok
Exercice.
\end{proof}

\begin{definition}
    \label{MyPrerat.inv}
    \lean{MyPrerat.inv}
    \leanok
    \uses{MyPrerat.R}
We define an operation, called \emph{negation} on $\MyPrerat$ as follows: the inverse of $x = (a,b)$ is:
\[
\mbox{if} b \neq 0 \mbox{ then } x^{-1} = (b, a), \mbox{ otherwise } x^{-1} = (0, 1)
\]
Note that $x^{-1}$ is \emph{always} defined!
\end{definition}

\begin{lemma}
    \label{MyPrerat.inv_quotient}
    \lean{MyPrerat.inv_quotient}
    \leanok
If $x \approx x'$, then $x^{-1} \approx x'^{-1}$.
\end{lemma}
\begin{proof}
\uses{MyPrerat.inv}
\leanok
Exercice.
\end{proof}

\section{The rationals}

\subsection{Definitions}

\begin{definition}
    \label{MyRat}
    \lean{MyRat}
    \leanok
    \uses{MyPrerat.R_equiv}
    We define our rationals $\MyRat$ as
\[
\MyRat = \MyPrerat \, / \approx
\]
We will write $⟦ (a, b) ⟧$ for the class of $(a,b)$.
\end{definition}

\begin{definition}
    \label{MyRat.zero}
    \lean{MyRat.zero}
    \uses{MyPrerat.R_equiv}
    \leanok
We define the zero of $\MyRat$, denoted $0$ as the class of $(0,1)$ (note that $1 \neq 0$ in $\MyInt$).
\end{definition}

\begin{definition}
    \label{MyRat.one}
    \lean{MyRat.one}
    \uses{MyPrerat.R_equiv}
    \leanok
We define the one of $\MyRat$, denoted $1$ as the class of $(1,1)$ (note that $1 \neq 0$ in $\MyInt$).
\end{definition}

\subsection{Commutative ring structure}

\begin{definition}
    \label{MyRat.neg}
    \lean{MyRat.neg}
    \uses{MyPrerat.R_equiv, MyPrerat.neg_quotient}
    \leanok
We define the negation of $x = ⟦ (a, b) ⟧$ in $\MyInt$ as
\[
-x = ⟦ -(a, b) ⟧
\]
Thanks to Lemma~\ref{MyPrerat.neg_quotient} this is well defined.
\end{definition}

\begin{definition}
    \label{MyRat.add}
    \lean{MyRat.add}
    \uses{MyPrerat.R_equiv, MyPrerat.add_quotient}
    \leanok
We define the addition of $x = ⟦ (a, b) ⟧$ and $y = ⟦ (c, d) ⟧$ in $\MyInt$ as
\[
x + y = ⟦ (a,b)+(c,d) ⟧
\]
Thanks to Lemma~\ref{MyPrerat.add_quotient} this is well defined.
\end{definition}

\begin{definition}
    \label{MyRat.mul}
    \lean{MyRat.mul}
    \uses{MyPrerat.R_equiv, MyPrerat.mul_quotient}
    \leanok
We define the multiplication of $x = ⟦ (a, b) ⟧$ and $y = ⟦ (c, d) ⟧$ in $\MyInt$ as
\[
x * y = ⟦ (a, b)*(c,d) ⟧
\]
Thanks to Lemma~\ref{MyPrerat.mul_quotient} this is well defined.
\end{definition}

\begin{definition}
    \label{MyRat.inv}
    \lean{MyRat.inv}
    \uses{MyPrerat.R_equiv, MyPrerat.inv_quotient}
    \leanok
We define the negation of $x = ⟦ (a, b) ⟧$ in $\MyInt$ as
\[
x^{-1} = ⟦ (a, b)^{-1} ⟧
\]
Thanks to Lemma~\ref{MyPrerat.inv_quotient} this is well defined.
\end{definition}

\begin{proposition}
    \label{MyRat.commRing}
    \lean{MyRat.commRing}
    \leanok
    $\MyRat$ with addition and multiplication is a commutative ring.
\end{proposition}
\begin{proof}
    \uses{MyRat.add, MyRat.mul, MyRat.zero, MyRat.one}
    \leanok
    We have to prove various properties, namely:
    \begin{itemize}
        \item addition is associative
        \item $0$ works as neutral element for addition (on both sides)
        \item existence of an inverse for addition (we prove that $x + (-x) = (-x) + x = 0$)
        \item addition is commutative
        \item left and right distributivity of multiplication with respect to addition
        \item associativity of multiplication
        \item $1$ works as neutral element for multiplication (on both sides)
    \end{itemize}
    All the proofs are essentially identical, going to $\MyInt$, unravelling the definition and then
    checking the equality holds in $\MyInt$.
\end{proof}

\begin{lemma}
    \label{MyRat.zero_ne_one}
    \lean{MyRat.zero_ne_one}
    \leanok
In $\MyRat$ we have $0 \neq 1$.
\end{lemma}
\begin{proof}
    \leanok
    \uses{MyRat.zero, MyRat.one}
    If $0 = 1$ by definition we would have $⟦ (0,1) ⟧ = ⟦ (1,1) ⟧$
    so $0*1=1*0$ in $\MyInt$, that is absurd.
\end{proof}

\begin{lemma}
    \label{MyRat.mul_inv_cancel}
    \lean{MyRat.mul_inv_cancel}
    \leanok
Let $x \neq 0$ be in $\MyRat$. Then $x * x^{-1} = 1$.
\end{lemma}
\begin{proof}
    \leanok
    \uses{MyRat.inv, MyRat.mul}
    Let $x = ⟦ (a,b) ⟧$, with $b \neq 0$. Since $x \neq 0$ we have $a \neq 0$ and so
    $x^{-1} = ⟦ (b,a) ⟧$. The lemma follows by definition of multiplication.
\end{proof}

\begin{proposition}
    \label{MyRat.field}
    \lean{MyRat.field}
    \leanok
    $\MyRat$ with addition and multiplication is a field.
\end{proposition}
\begin{proof}
    \uses{MyRat.commRing, MyRat.mul_inv_cancel}
    \leanok
    Clear because of Lemma~\ref{MyRat.mul_inv_cancel}.
\end{proof}

\subsection{\texorpdfstring{The inclusion $i \colon \N \to \MyRat$}{The inclusion}}
\begin{definition}
    \label{MyRat.i}
    \lean{MyRat.i}
    \uses{MyRat}
    \leanok
    We define a map
\begin{gather*}
    i \colon \N \to \MyRat \\
    n \mapsto ⟦ (\MyInt.i n,1) ⟧
\end{gather*}
\end{definition}

\begin{lemma}
    \label{MyRat.i_zero}
    \lean{MyRat.i_zero}
    \leanok
We have that $i(0) = 0$.
\end{lemma}
\begin{proof}
    \uses{MyRat, MyRat.i}
    \leanok
Clear from the definition.
\end{proof}

\begin{lemma}
    \label{MyRat.i_one}
    \lean{MyRat.i_one}
    \leanok
We have that $i(1) = 1$.
\end{lemma}
\begin{proof}
    \uses{MyRat, MyRat.i}
    \leanok
Clear from the definition.
\end{proof}

\begin{lemma}
    \label{MyRat.i_add}
    \lean{MyRat.i_add}
    \leanok
For all $a$ and $b$ in $\N$ we have that
\[
i(a+b) = i(a) + i(b)
\]
\end{lemma}
\begin{proof}
    \uses{MyRat.add, MyRat.i}
    \leanok
    We have $i(a+b) = ⟦ (\MyInt.i (a+b), 1) ⟧ = ⟦ (\MyInt.i (a), 1) + (\MyInt.i (b), 1) ⟧$, $i(a) = ⟦ (\MyInt.i(a), 1) ⟧$ and
    $i(b) = ⟦ (\MyInt.i (b), 1) ⟧$, so we need to prove that
\[
⟦ (\MyInt.i (a), 1) + (\MyInt.i (b), 1) ⟧ = ⟦ (\MyInt.i (a), 1) ⟧ + ⟦ (\MyInt.i (b), 1) ⟧
\]
that is obvious from the definition ().
\end{proof}

\begin{lemma}
    \label{MyRat.i_mul}
    \lean{MyRat.i_mul}
    \leanok
For all $a$ and $b$ in $\N$ we have that
\[
i(a*b) = i(a) * i(b)
\]
\end{lemma}
\begin{proof}
    \uses{MyRat.mul, MyRat.i}
    \leanok
Similar to the proof of Lemma~\ref{MyRat.i_add}.
\end{proof}

\begin{lemma}
    \label{MyRat.i_injective}
    \lean{MyRat.i_injective}
    \leanok
    We have that $i$ is injective.
\end{lemma}
\begin{proof}
    \uses{MyRat.i}
    \leanok
    Let $a$ be such that $i(a)=0$. This means $⟦ (\MyInt.i a,1) ⟧ = ⟦ (0,1) ⟧$ so $(\MyInt.i a) * 1 = 1 * 0$ and hence $\MyInt.i a = 0$ so $a = 0$.
\end{proof}

\subsection{\texorpdfstring{The inclusion $j \colon \MyInt \to \MyRat$}{The inclusion}}
\begin{definition}
    \label{MyRat.j}
    \lean{MyRat.j}
    \uses{MyRat}
    \leanok
    We define a map
\begin{gather*}
    j \colon \MyInt \to \MyRat \\
    n \mapsto ⟦ (n,1) ⟧
\end{gather*}
\end{definition}

\begin{lemma}
    \label{MyRat.j_zero}
    \lean{MyRat.j_zero}
    \leanok
We have that $j(0) = 0$.
\end{lemma}
\begin{proof}
    \uses{MyRat, MyRat.j}
    \leanok
Clear from the definition.
\end{proof}

\begin{lemma}
    \label{MyRat.j_one}
    \lean{MyRat.j_one}
    \leanok
We have that $j(1) = 1$.
\end{lemma}
\begin{proof}
    \uses{MyRat, MyRat.j}
    \leanok
Clear from the definition.
\end{proof}

\begin{lemma}
    \label{MyRat.j_add}
    \lean{MyRat.j_add}
    \leanok
For all $a$ and $b$ in $\MyInt$ we have that
\[
j(a+b) = j(a) + j(b)
\]
\end{lemma}
\begin{proof}
    \uses{MyRat.add, MyRat.j}
    \leanok
Exercice.
\end{proof}

\begin{lemma}
    \label{MyRat.j_mul}
    \lean{MyRat.j_mul}
    \leanok
For all $a$ and $b$ in $\N$ we have that
\[
j(a*b) = j(a) * j(b)
\]
\end{lemma}
\begin{proof}
    \uses{MyRat.mul, MyRat.j}
    \leanok
Exercice.
\end{proof}

\begin{lemma}
    \label{MyRat.j_injective}
    \lean{MyRat.j_injective}
    \leanok
    We have that $j$ is injective.
\end{lemma}
\begin{proof}
    \uses{MyRat.j}
    \leanok
Exercice.
\end{proof}

\begin{lemma}
    \label{MyRat.j_comp_eq_i}
    \lean{MyRat.j_comp_eq_i}
    \leanok
    Let $n$ be a natural number. Then $\MyRat.j (\MyInt.i (n)) = MyRat.i (n)$.
\end{lemma}
\begin{proof}
    \uses{MyRat.i, MyRat.j}
    \leanok
It follows from unravelling all the definitions.
\end{proof}

\begin{lemma}
    \label{MyRat.Quotient.mk_def}
    \lean{MyRat.Quotient.mk_def}
    \leanok
    Let $a$ and $b$ be in $\MyInt$ with $b \neq 0$. Then $⟦ (a, b) ⟧ = j(a)*j(b)^{-1}$.
\end{lemma}
\begin{proof}
    \uses{MyRat.inv, MyRat.mul, MyRat.j}
    \leanok
Exercice.
\end{proof}

\subsection{Nonnegativity}

Before defining the order on $\MyRat$, let's define the notion of nonnegativity.

\begin{definition}
    \label{MyRat.IsNonneg}
    \lean{MyRat.IsNonneg}
    \uses{MyRat, MyRat.zero}
    \leanok
Given $x = (a,b)$ in $\MyRat$, we say that $x$ is \emph{nonnegative} if $0 \leq a$ and $0 < b$.

Can you see why it corresponds to the ``usual definition'' when we think that $x = a/b$?
\end{definition}

\begin{lemma}
    \label{MyRat.zero_nonneg}
    \lean{MyRat.zero_nonneg}
    \uses{MyRat.IsNonneg}
    \leanok
    We have that $0$ in $\MyRat$ is nonnegative.
\end{lemma}
\begin{proof}
    \leanok
    Obvious.
\end{proof}

\begin{lemma}
    \label{MyRat.one_nonneg}
    \lean{MyRat.one_nonneg}
    \uses{MyRat.IsNonneg, MyRat.one}
    \leanok
    We have that $1$ in $\MyRat$ is nonnegative.
\end{lemma}
\begin{proof}
    \leanok
    Obvious.
\end{proof}

\begin{lemma}
    \label{MyRat.nonneg_neg}
    \lean{MyRat.nonneg_neg}
    \uses{MyRat.IsNonneg, MyRat.neg}
    \leanok
    Let $x$ be in $\MyRat$ such that both $x$ and $-x$ are nonnegative. Then $x = 0$.
\end{lemma}
\begin{proof}
    \leanok
    Unravelling all the definitions we end up with $a$, $b$, $c$ and $d$ in $\MyInt$ such that $0 \leq a$, $0 < b$, $0 \leq c$, $0 < d$ and $-(a*d)=b*c$. This implies $a=0$.
\end{proof}

\begin{lemma}
    \label{MyRat.nonneg_neg_of_not_nonneg}
    \lean{MyRat.nonneg_neg_of_not_nonneg}
    \uses{MyRat.IsNonneg, MyRat.neg}
    \leanok
    Let $x$ be in $\MyRat$ such that $x$ is not nonnegative. Then $-x$ is nonnegative.
\end{lemma}
\begin{proof}
    \leanok
    Annoying but easy, left as an exercice.
\end{proof}

\begin{lemma}
    \label{MyRat.isNonneg_add_isNonneg}
    \lean{MyRat.isNonneg_add_isNonneg}
    \uses{MyRat.IsNonneg, MyRat.add}
    \leanok
    Let $x$ and $y$ be in $\MyRat$ both nonnegative. Then $x+y$ is nonnegative.
\end{lemma}
\begin{proof}
    \leanok
    Exercice.
\end{proof}

\begin{lemma}
    \label{MyRat.isNonneg_mul_isNonneg}
    \lean{MyRat.isNonneg_mul_isNonneg}
    \uses{MyRat.IsNonneg, MyRat.mul}
    \leanok
    Let $x$ and $y$ be in $\MyRat$ both nonnegative. Then $x*y$ is nonnegative.
\end{lemma}
\begin{proof}
    \leanok
    Exercice.
\end{proof}

\begin{lemma}
    \label{MyRat.isNonneg_inv_isNonneg}
    \lean{MyRat.isNonneg_inv_isNonneg}
    \uses{MyRat.IsNonneg, MyRat.inv}
    \leanok
    Let $x$ be in $\MyRat$ be nonnegative. Then $x^{-1}$ is nonnegative.
\end{lemma}
\begin{proof}
    \leanok
    Exercice.
\end{proof}

\subsection{The order}

\begin{definition}
    \label{MyRat.le}
    \lean{MyRat.le}
    \uses{MyRat.IsNonneg, MyRat.field}
    \leanok
Let $x$ and $y$ in $\MyRat$. We write $x \leq y$ if $y - x$ is nonnegative.
\end{definition}

\begin{lemma}
    \label{MyRat.zero_le_iff_IsNonneg}
    \lean{MyRat.zero_le_iff_IsNonneg}
    \uses{MyRat.zero, MyRat.le}
    \leanok
    We have that $0 \leq x$ if and only if $x$ is nonnegative.
\end{lemma}
\begin{proof}
    \leanok
    Clear.
\end{proof}

\begin{lemma}
    \label{MyRat.zero_le_one}
    \lean{MyRat.zero_le_one}
    \leanok
    In $\MyRat$ we have that $0 \leq 1$.
\end{lemma}
\begin{proof}
    \uses{MyRat.zero, MyRat.one_nonneg}
    \leanok
    Clear because of Lemma~\ref{MyRat.one_nonneg}.
\end{proof}

\begin{lemma}
    \label{MyRat.le_refl}
    \lean{MyRat.le_refl}
    \leanok
    The relation $\leq$ on $\MyRat$ is reflexive.
\end{lemma}
\begin{proof}
    \uses{MyRat.le, MyRat.zero_nonneg}
    \leanok
    Clear because of Lemma~\ref{MyRat.zero_nonneg}.
\end{proof}

\begin{lemma}
    \label{MyRat.le_trans}
    \lean{MyRat.le_trans}
    \leanok
    The relation $\leq$ on $\MyRat$ is transitive.
\end{lemma}
\begin{proof}
    \uses{MyRat.isNonneg_add_isNonneg}
    \leanok
 It follows from Lemma~\ref{MyRat.isNonneg_add_isNonneg}.
\end{proof}

\begin{lemma}
    \label{MyRat.le_antisymm}
    \lean{MyRat.le_antisymm}
    \leanok
    The relation $\leq$ on $\MyRat$ is antisymmetric.
\end{lemma}
\begin{proof}
    \uses{MyRat.nonneg_neg}
    \leanok
    It follows from Lemma~\ref{MyRat.nonneg_neg}.
\end{proof}

It follows that $\leq$ is an order relation.


\subsection{Interaction between the order and the ring structure}

\begin{lemma}
    \label{MyRat.add_le_add_left}
    \lean{MyRat.add_le_add_left}
    \leanok
    Let $x$, $y$ and $z$ in $\MyRat$ be such that $x \leq y$. Then $z + x ≤ z + y$.
\end{lemma}
\begin{proof}
    \uses{MyRat.le, MyRat.add}
    \leanok
    Clear from the definitions.
\end{proof}

\begin{lemma}
    \label{MyRat.mul_nonneg}
    \lean{MyRat.mul_nonneg}
    \leanok
    Let $x$ and $y$ in $\MyRat$ be such that $0 \leq x$ and $0 \leq y$. Then $0 \leq x * y$.
\end{lemma}
\begin{proof}
    \leanok
    \uses{MyRat.isNonneg_mul_isNonneg}
    It follows from Lemma~\ref{MyRat.isNonneg_mul_isNonneg}.
\end{proof}

We have proved that $\MyRat$ is an ordered ring.

\subsection{Interaction between the order and the inclusions}

\begin{lemma}
    \label{MyRat.j_le_iff}
    \lean{MyRat.j_le_iff}
    \leanok
    Let $x$ and $y$ in $\MyInt$. We have that $j(x) \leq j(y)$ if and only if $x \leq y$.
\end{lemma}
\begin{proof}
    \uses{MyRat.IsNonneg, MyInt.le}
    \leanok
    Exercice.
\end{proof}

\begin{lemma}
    \label{MyRat.i_le_iff}
    \lean{MyInt.i_le_iff}
    \leanok
    Let $x$ and $y$ in $\N$. We have that $i(x) \leq i(y)$ if and only if $x \leq y$.
\end{lemma}
\begin{proof}
    \uses{MyRat.IsNonneg, MyInt.i_le_iff, MyRat.j_le_iff, MyRat.j_comp_eq_i}
    \leanok
    It follows immediately by Lemma~\ref{MyInt.i_le_iff}, Lemma~\ref{MyRat.j_le_iff} and Lemma~\ref{MyRat.j_comp_eq_i}.
\end{proof}

\subsection{The linear order structure}

\begin{lemma}
    \label{MyRat.le_total}
    \lean{MyRat.le_total}
    \leanok
    The order $\leq$ on $\MyRat$ is a total order.
\end{lemma}
\begin{proof}
    \uses{MyRat.nonneg_neg_of_not_nonneg}
    \leanok
This follows by Lemma~\ref{MyRat.nonneg_neg_of_not_nonneg}.
\end{proof}

\begin{lemma}
    \label{MyRat.linearOrder}
    \lean{MyRat.linearOrder}
    \leanok
    We have that $\MyRat$ with $\leq$ is a \emph{linear order}
\end{lemma}
\begin{proof}
    \uses{MyRat.le_refl, MyRat.le_antisymm, MyRat.le_trans, MyRat.le_total}
    \leanok
    Clear from the lemma above.
\end{proof}

\begin{lemma}
    \label{MyRat.mul_pos}
    \lean{MyRat.mul_pos}
    \leanok
    Let $x$ and $y$ in $\MyRat$ be such that $0 < x$ and $0 < y$. Then $0 < x * y$.
\end{lemma}
\begin{proof}
    \leanok
    \uses{MyRat.zero_le_iff_IsNonneg, MyRat.isNonneg_mul_isNonneg}
Exercice.
\end{proof}

We now have that $\MyRat$ is a \emph{strict ordered ring}: a nontrivial ring with a partial order such that addition is strictly monotone and multiplication by a positive number is strictly monotone.